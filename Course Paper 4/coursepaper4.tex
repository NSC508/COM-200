\documentclass[a4paper]{article}
\usepackage[utf8]{inputenc}
\usepackage[english]{babel} 
\usepackage{hyperref} 
\usepackage{float}
\usepackage{amsmath} 
\usepackage{graphicx}
\usepackage[colorinlistoftodos]{todonotes} 
\usepackage{tikz}
\usepackage{pdfpages} 
\usepackage{listings}
\usepackage[margin=1in]{geometry}
\usepackage{natbib}
\usepackage{mathptmx}
\usepackage{setspace}

\title {
	Course Paper 4
}

\author {
	\normalsize Sathvik Chinta\\\normalsize
    \normalsize Communication 200\\\normalsize
}

\date {
	\color{black} December 11, 2022
}

\doublespacing
\begin{document} \maketitle
    \setcitestyle{authoryear,open={(},close={)}}
    \section{}
        \textbf{1. The “actual malice” standard that the US uses to determine defamation has come under fire recently. This summer, the U.S. Supreme Court refused to take a defamation case, despite some Justices wanting to take on the case. Here’s a short article explaining the basics of the case.  Links to an external site.Your answer should explain what defamation is and why public officials need to show actual malice. Next, explain why the “War Dogs” case fits into the existing definition. Finally, explain why you agree/disagree with Justices Thomas and Gorsuch’s calls to change the “actual malice” standard.}

        In lecture, we defined defamation as “a false statement that is communicated to third persons and causes harm to the 
        reputation of the person who was the subject” \citep{lesson9}. There are two main forms of defamation that we discussed 
        in lecture: libel and slander. Libel defamation examples are in permanent form, such as present in written pieces, drawings, 
        etc. Slander defamation cases are only present in transitory forms, such as in oral communication. \citep{lesson9}. 
        Public officials are, by nature, in positions of tremendous power. As such, one could say that any form of criticism against 
        the official and their policies can be seen as a form of defamation. However, this is a right that is protected by the 
        First Amendment. To get through this impasse, public officials must prove actual malice in whoever they’re sueing for 
        defamation.
        
        War dogs fits perfectly into this description. “Shkelzen Berisha, the son of Albania's former prime minister, sued 
        author Guy Lawson for defamation over his book” \citep{williams_2021}. The main issue that Berisha has with the story, 
        however, is how he’s detailed as being a key member in the Albanian mafia: “As the plot unfolded, the three 
        [main characters] have run-ins with the Albanian mafia, describing Berisha as a key figure in the organization.” 
        \citep{williams_2021}. Berisha, denying all involvement with the Albanian mafia, attempted to sue on charges of defamation. 
        
        This case fits into the definition of defamation quite well. Being a high-ranking official (as son of Albania’s former 
        prime minister), being associated with the mafia, and being described as a key figure of the organization at that, would
        make the public very weary to interact with them in the future. Furthermore, this was a book that then became a film which 
        are both permanent forms. As such, this is a libel defamation case.
        
        Justices Thomas and Gorsuch have argued that the "actual malice" standard should be changed, but their specific proposed 
        changes are not mentioned in the article. It is not clear whether I would agree or disagree with these proposed changes 
        without knowing more about them. However, this is a very sensitive issue. With too much power, political officials could 
        easy suppress free speech and claim that any post on social media is defamation. 


    \section{}
        \textbf{2. Define and explain the cultural imperialism argument as discussed in the lecture. Next summarize a recent (since 2018) example of co-produced Western cultural imperialism. Finally, point out a possible cultural implication of Western media presence.}
        
        In lecture we defined the cultural imperialism argument as “[one that] suggests that a large volume of media products flow from 
        the West, especially the US, and thus shape other cultures in a form of domination.” \citep{lesson10}. One clear example of this 
        is Hollywood. Hollywood cinema makes up a significant market share of the world’s movie markets, and as such it shapes 
        “cultural norms in the global scale with US centered values” \citep{lesson10}. 
        
        In the lecture, we also defined co-production as “mutual cooperation between international media companies in terms of capital 
        investment and the sharing of resources” \citep{lesson10}. 
        
        One example of co-produced Western cultural imperialism is the rise of K-Pop in the global music market. K-Pop, which originated
        in South Korea, has become increasingly popular in Western countries, with many Western artists collaborating with K-Pop bands and
        adopting K-Pop style and aesthetics in their own music. “Considering the growing international interest in K-pop, certainly it
        cannot be simply regarded as Korean popular music… both a product of transnational practices and a global term referring comprehensively
        to not simply Korean popular music produced by the Korean music industry and consumed overseas, but to related cultural phenomenon
        as well” \citep{phuong_2016}. This has been facilitated by co-production between Western and South Korean media companies, with both
        sides investing capital and sharing resources in order to produce and promote K-Pop music. “To account for the international success
        of K-pop, one cannot fail to mention the 'not so invisible hand' of the South Korean government that fosters the happy marriage between
        K-pop and other media/cultural industries… it has continuously been blessed with state power in the forms of capital investment, tax
        benefits and support for overseas expansion” \citep{phuong_2016}. This co-production has contributed to the spread of South Korean 
        cultural practices, such as the use of Korean words and phrases, to Western audiences, while at the same time, 
        Western cultural practices are being adopted by K-Pop bands. 
        
        One possible cultural implication of Western media presence is the homogenization of cultural practices, as 
        dominant Western cultural practices spread to other societies and potentially replace their own unique cultural practices. 
        This can lead to the loss of cultural diversity and the erosion of cultural identities.


    \section{}
        \textbf{3. Define and explain the global counter flow argument as discussed in the lecture. Next summarize a recent (since 2018) example of global counter flow. Finally, point out a limitation of global counter flow argument in relation to the discussion of global media ownership.}
        
        In lecture, we defined the global counter flow as “the hypothesis which shows that globalization of media, people, and technology enabled for previously 
        marginalized global south to counter circulate and distribute their own messages to the rest of the world” \citep{lesson10}. In its most basal sense, 
        it is an opposite to the western cultural imperialism argument. 
        
        One example of a recent global counter flow is Indian cinema. I myself am from Hyderabad, the center of Telugu cinema, and as such I’ve watched countless 
        Tollywood (basically the Telugu version of Bollywood) movies throughout my lifetime. My favorite movie with my favorite actor, and a movie discussed in class, 
        is Baahubali. This was directed by S. S. Rajamouli, a big name in the Tollywood industry. This film went above and beyond the previous standards seen in any 
        Indian cinema, as it created a high fantasy universe filled with tribes, kingdoms, and storylines of love, betrayal, and sadness. 
        
        The film itself was released simultaneously in many of the Indian languages in addition to its original Telugu, and Netflix soon bought the streaming rights 
        to the movie bringing Baahubali to the global stage. It remains as one of the highest grossing Indian films of all time, with it’s sequel (Baahubali 2) being 
        the highest grossing domestic Indian film of all time. It brought not only Telugu films, but Indian films as a whole to a much higher level and accelerated the 
        spreading of Indian culture, ideals, and methodologies to the rest of the world. S.S. Rajamouli’s next film, RRR (also seen in lecture), was just as influential 
        as it told a freedom-fighting story about the British occupation of India and two freedom fighters doing their part in the rebellion. This, and future movies on 
        this scale from Tollywood, are set to propel Indian cinemas to the forefront of the media world. \citep{lesson10}
        
        One limitation to the global counterflow argument in relation to the discussion of global media ownership is that it currently does not consider two major issues. 
        Firstly, scale. Even though Indian cinema produces the most movies in the world per year, their box office numbers pale in comparison to Hollywood. So, whilst 
        Baahubali set new record for the Indian box office, it still doesn’t hold a candle to American cinema’s gargantuan numbers. Secondly, pure global counter flows 
        are rare in the world. Often times, gigantic efforts such as movies are funded and have influence from all over the world. Baahubali adopted many western media film 
        practices, in addition to gaining sponsorships from American companies such as AMD for the CGI in the movies. As a result, the global media landscape may not be as 
        diverse and varied as the global counterflow argument suggests.


    \section{}
        \textbf{4. Find a speech or written editorial and analyze it in terms of its appeals to logos, pathos, and ethos. In your answer, you should summarize the speech or text and provide a link. Define logos and show where the text includes appeals to logos. Define pathos and show where the text includes appeals to pathos. Define ethos and show where the text includes appeals to ethos. Finally, render a judgment. Was this particular mixture of logos, pathos, and ethos effective or not?}

        \pagebreak
        \bibliographystyle{apalike}
        \bibliography{myrefs}
        \cite{lesson9}
        \cite{williams_2021}
        \cite{lesson10}
        \cite{phuong_2016}

\end{document}

