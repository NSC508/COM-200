\documentclass[a4paper]{article}
\usepackage[utf8]{inputenc}
\usepackage[english]{babel} 
\usepackage{hyperref} 
\usepackage{float}
\usepackage{amsmath} 
\usepackage{graphicx}
\usepackage[colorinlistoftodos]{todonotes} 
\usepackage{tikz}
\usepackage{pdfpages} 
\usepackage{listings}
\usepackage[margin=1in]{geometry}
\usepackage{natbib}
\usepackage{mathptmx}
\usepackage{setspace}

\title {
	Course Paper 3
}

\author {
	\normalsize Sathvik Chinta\\\normalsize
    \normalsize Communication 200\\\normalsize
}

\date {
	\color{black} November 20, 2022
}

\doublespacing
\begin{document} \maketitle
    \setcitestyle{authoryear,open={(},close={)}}
    \section{}
        \textbf{1. Following Week 7 lecture, first, define media agenda building OR media framing. Now, find an example from a published online news media article about the most recent midterm elections, and explain how the theoretical concepts are represented in the example you’ve chosen. For instance, you might choose a particular senate race (such as: Arizona Senate, Florida or Texas governor) or a significant campaign / topic (for instance: abortion, healthcare, migration, tax, job, policing, inflation and prices, climate etc.), then, look for a news article about that campaign / topic that exemplifies your chosen concept. Note that the midterm campaigns have now concluded, but your example can be from any point in a campaign. Make sure to cite the news article using in-text citation and give a URL link of the news article in APA format in the bibliography (https://wr1ter.com/how-to-cite-a-newspaper-article-in-apa). }

        Media Framing is a harsh truth in the world of media. There is no such thing as “unbiased media”, as every corporation, publication, and author will have their own ways of viewing certain issues and those viewpoints will inevitably leak onto the page. 
        In the lecture, we defined framing as “select[ing] some aspects of a perceived reality and make them more salient in a communication text, in such a way as to promote a particular problem definition, causal interpretation, moral evaluation and or/treatment 
        recommendation for the item described” \citep{lesson7}. News sources are the ones in charge of dictating what you read, and while they can’t tell you how to perceive it, they can list the facts in a certain viewpoint to make you believe you reached the 
        conclusion on our own. 

        I chose an article from Vox. Vox is owned by Vox Media, and Comcast has a majority stake (~34\%) in Vox Media, so we can already understand that there might be bias in these articles to have viewpoints that align with Comcast’s agenda. 
        The article is titled “Democrats hang on to their crucial Senate majority” and it's written by Li Zhou. In the article, they speak about how the Democrats have kept a senate majority and the effects of this. The article begins with a seemingly neutral stance, 
        speaking of how the democratic party has secured at least 50 seats, and “Should the party prevail in Georgia, they’ll have ended the midterms with a net gain of one seat.” \citep{vox}. However, as the article goes on to talk about what the Democrats 
        can accomplish with their senate majority, they begin to speak about all the positives of this victory. They also speak about how if the Democrats were to win the House, Democrats would be able make “improvements to the Affordable Care Act, new climate regulations, 
        as well as child care, educational, and tax policies that didn’t make it into Democrats’ Inflation Reduction Act.” \citep{vox}. 

        The main takeaway from reading this article is how pro-Democrat it is. I did a little bit of research into Vox, and found that they are described as a liberal media group, explaining their bias. We learned in lecture that “a frame focus[es] on certain values over 
        others.” \citep{lesson7}. The article focuses on Democrat-centric values much more than republican, and doesn’t even delve into what this result means for the republicans. 


    \section{}
        \textbf{3. Find a public opinion poll published within the past few months. You can find these anywhere, but https://www.pewresearch.org/ Links to an external site.is pretty good about letting to see the methodology.}
        
        I chose a poll titled “Most across 19 countries see strong partisan conflicts in their society, especially in South Korea and the U.S.” (https://www.pewresearch.org/fact-tank/2022/11/16/most-across-19-countries-see-strong-partisan-conflicts-in-their-society-especially-in-south-korea-and-the-u-s/ ). 
        This poll was run by the Pew Research Center and reported on by Laura Silver, a senior researcher at the Pew Research Center. This poll asked participants two main questions. Firstly, they asked about conflict between people who support varying political parties from their home country. 
        Participants were asked to categorize the conflicts as very strong, strong, not very strong, or no conflicts. Then, they were asked the same question, but about the United States as the subject instead of their home country. 

        For data from foreign countries, the study was conducted between February 14 to June 3 2022. In all countries except for Hungary, Poland, and Isreal (where the surveys were conducted face-to-face) and in Australia (where it was conducted online), the surveys were conducted over the phone.
        The total number of foreign adults totaled to 20,944. In the United States, the Pew Research Center surveyed 3,581 adults from March 21 to March 27 2022. Only those who were members of the Center’s American Trends Panel took the survey, which is recruited through random sampling of addresses.
        In addition to this, the survey is weighted for “gender, race, ethnicity, partisan affiliation, education, and other categories” \citep{poll}. 

        In the lecture, we learned that margin of error represents the “range of possible results” \citep{lesson7}. Often times, even with random sampling, the exact value for the data isn’t exactly equal to whatever the survey results say. The margin of error allows us to 
        say “the actual result is somewhere within this range”, so as to not be misleading. Each country in the survey had their own margin of error reported. These ranged from as low as 2.8 percentage points (Australia) to as high as 4.4 (Poland). 

        In the lecture, we learned that the confidence level is “what would it look like if you interviewed the entire population” \citep{lesson7}. A sample, by definition, will never encompass the entirety of a population. As a result, you can never accurately claim that your 
        sample survey is indicative of an entire population. That’s where confidence level comes in. If you had a 97\% confidence level, with a 2\% margin of error, the answer would fall within 2\% of your survey’s results 97\% of the time. This particular survey reported their margins of 
        error with a 95\% confidence level. 

        This poll has many qualities that make it a good poll. It has a very high confidence level with a comparatively low margin of error for each country. The only concern with this poll is their sampling methods. Phone calls have clear sampling bias, as only participants that are willing
         to answer their phone and speak to the surveyors will be represented in the survey. This might not be indicative of the entire population as a result, and the survey results can only apply to this smaller subset of the overall demographic. 


    \section{}
        \textbf{4. Explain a major case of media and/or telecommunication ownership concentration in the political economy approach taking a parent company as identified in the lecture. Explain with examples how the parent company is vertically and horizontally integrated and whether it has implications for the public interest. Finally, do you think the FCC should regulate such concentration of ownership? Why or why not?}
        
        Comcast is a major company in the American landscape. They are a gigantic telecommunications company that own companies in many industries, such as Universal Studios, NBC, Xfinity, a large stake in Hulu.
        Each of these companies belong to different sectors of industry. For instance, Xfinity is an ISP as well as a cable company. Hulu is a streaming service, while Universal Studios makes movies.
        NBC is one of the largest and oldest media companies in the United States, and being owned by a company that is an internet service provider as well is alarming. 

        In lecture, we define vertical integration as “controlling production, distribution, and exhibition” \citep{lesson8}. Since Comcast owns companies at every level of media, they are the ones that get to control 
        the production, distribution, and execution of the media. We also defined horizontal integration as “purchasing other outlets at the same level of production”. Comcast owns USA Networks, E!, NBC, and other TV channels, 
        all of which show programs on the daily. Consumers are given the notion that they have free will and that they are the ones choosing which channel to watch to diversify their viewpoints, but when all the channels are 
        owned by the same company, those viewpoints often coincide with each other and don’t paint the full picture. In lecture, we discussed the idea of framing as “select[ing] some aspects of a perceived reality and make 
        them more salient in a communication text, in such a way as to promote a particular problem definition, causal interpretation, moral evaluation and or/treatment recommendation for the item described” \citep{lesson8}. 
        Comcast, like any other large corporation, has the final goal of making as much money as possible. As such, they adhere to the Media Ownership filter we learned in class from the video by Noam Chomsky. Media is owned by 
        big corporations, who only care about profit. As a result, they will naturally push whatever agenda gets them more profit. \citep{video}

        The FCC should definitely be wary and regulate corporations like Comcast. Comcast owns both NBC (a news channel) and Xfinity (an ISP). If Comcast were able to have their way, then another Media Ownership filter 
        would apply: Advertising Money. If a certain advertiser (a website for instance) were to pay Comcast, Comcast could increase the amount of bandwidth allocated for its users to go to that website whilst stunting 
        the rest. The end result would be smaller loading times when accessing that particular website, meaning that users would be more likely to visit that website as opposed to its competitors, resulting in monopolistic 
        practices. 


    \section{}
        \textbf{5. Explain the concept of net-neutrality in political economy approach. Explain whether repealing net neutrality affects public interest. Finally, do you think the FCC should uphold net neutrality policy? Why or why not? }
        
        We learned in lecture that Net Neutrality, in it’s basic sense, is restricting ISPs to not be discriminatory towards certain websites or applications. Take Comcast for example. They own Xfinity, one of the largest ISPs 
        in the United States. Without Net Neutrality, Comcast would be allocate dynamic amounts of bandwidth based on which applications or websites they want you to visit. For instance, they would allow unlimited bandwidth 
        towards NBC (their news company), whilst limiting the amount of bandwidth towards other sources. 

        The five filters of media already apply to televised news sources. The five filters are media ownerships, advertising money, the media elite, flack, and the common enemy \citep{video}. These five allow the 
        media to be controlled and let the public be marginalized towards certain ideals and viewpoints based purely on whatever increases the media company’s bottom line. If net-neutrality were to be appealed, these 
        five filters would transition over to internet as well. We would see media consolidation and the collapse of the payment model, as we saw in class. In this scenario, many newspaper shops were bought out by the 
        Sinclair Broadcast Group, and all users eventually funneled towards one of the newspapers bought by the group. New newspapers had struggle gaining popularity in this market as a result of the widespread adaptation 
        of the media group, and the group themselves were able to push whatever narrative they so desired to the public \citep{lesson8}. 

        To battle the concepts of hiveminds, allow multiple viewpoints to be considered, and promote free discussion, then, I think that the FCC should uphold net neutrality policies. While media corporations would 
        lose out on money, the ultimate loser in this scenario would be us, the consumers. Imagine a world in which your new internet company was not able to take off as a result of Xfinity limiting your broadband traffic. 
        Even worse, imagine coming up with a brand new idea for a company, only for Xfinity to steal your idea, launch a rival company and block all traffic to your website in order to increase their own bottom line. 


        \pagebreak
        \bibliographystyle{apalike}
        \bibliography{myrefs}
        \cite{lesson7}
        \cite{lesson8}
        \cite{vox}
        \cite{poll}
        \cite{video}
\end{document}

