\documentclass[a4paper]{article}
\usepackage[utf8]{inputenc}
\usepackage[english]{babel} 
\usepackage{hyperref} 
\usepackage{float}
\usepackage{amsmath} 
\usepackage{graphicx}
\usepackage[colorinlistoftodos]{todonotes} 
\usepackage{tikz}
\usepackage{pdfpages} 
\usepackage{listings}
\usepackage[margin=1in]{geometry}
\usepackage{natbib}
\usepackage{mathptmx}
\usepackage{setspace}

\title {
	Course Paper 2
}

\author {
	\normalsize Sathvik Chinta\\\normalsize
    \normalsize Communication 200\\\normalsize
}

\date {
	\color{black} November 4, 2022
}

\doublespacing
\begin{document} \maketitle
    \setcitestyle{authoryear,open={(},close={)}}
    \section{}
        \textbf{Taking the example of EITHER Laboon practice by Mokens OR Haka performance by Māori people (not both), explain three characteristics of primary oral cultures aka orality-based thoughts of expression. Add, why the oral cultures like this continue to be significant in a society dominated by literate cultures.}

        The Haka performance by Māori people is a breathtaking example of primary oral cultures. 
        In their performances, the people demonstrate characteristics of being agonistically toned, 
        empathetic and participatory rather than objectively distanced, and conservative and 
        traditionalist. 

        The Haka performance is a perfect example of an agonistically toned oral culture. 
        “Many, if not all, oral or residually oral cultures strike literates as extraordinarily 
        agonistic in their verbal performance and indeed in their lifestyle” \citep{ong1982}. 
        The “War haka (peruperu) were originally performed by warriors before a battle, proclaiming 
        their strength and prowess in order to intimidate the opposition” \cite{hemo2015}. 
        Traditionally created as a war chant, the example we saw in the video showed a Haka 
        performance at a wedding. The war chant was directed towards the groom in this scenario, 
        and the message is clear; take care of the bride. The chant serves as a warning in addition
        to preserving the traditions of the Māori people

        The Haka performance is also empathetic and participatory. “For an oral culture learning 
        or knowing means achieving close, empathetic, communal identification with the known” 
        \citep{ong1982}. In the performance, there were a wide array of emotions on display.
        Many of the performers had threatening facial expressions, frequently putting their tongues 
        out in a form of intimidation. The performance heavily relied on loud, sudden sounds and 
        beats from the performers as well. The performance also elicited emotions from those watching, 
        as the bride as well as other members of the audience burst into tears during the Haka chant. 

        Finally, the Haka performance is conservative and traditionalist. The Haka chants are a 
        very old staple of New Zealand culture, with records showing the performance as far back as 
        1769 \citep{timoti1993}. As mentioned earlier, these were initially used to intimidate opponents
        before battle, but have been preserved throughout the years to be performed at multiple 
        functions. We saw an example in the lecture of the performance being enacted at a wedding, 
        but there are also examples of the Haka performance being used in sports. The New Zealand 
        men’s basketball team, for instance, used the war chant famously against the United States 
        team in the basketball world cup. The dance itself has changed little throughout the year,
        indicating its conservative status, while the messages and emotions conveyed also have 
        survived through the ages. 

    \section{}
        \textbf{Taking an example of Donald Trump's speech from the lecture, explain three characteristics of primary oral cultures aka orality-based thoughts of expression. Also, drawing from the analysis that we discussed in the lecture, provide a possible rationale why his speeches may not make sense to some but make perfect sense to others.}
        
        The Donald Trump speech from the lecture is a very interesting example of a primary oral culture. It shows the characteristics of being additive rather than subordinate, agonistically toned, and empathetic and participatory. 

        The Donald Trump speech may be the clearest form of additive rather than subordinate that we have seen. The entirety of the speech is one sentence, put together through multiple “and” statements. Heard all at once and without breaking down the individual phrases, which makes the speech very hard to understand. As a result, some listeners may have trouble following the thought process behind the speech as Trump goes through multiple tangents. Every phrase joined by “and” adds more to his overall argument, but sometimes the claims are not related to the last. As a result, the speech paints an overall picture for his argument yet the individual phrases only add diminishing returns to his final say. 

        The Donald Trump speech is also very empathetic. He goes into many anecdotes in order to prove his point, such as when he talks about his family member who went to MIT and is an expert in the field. He constantly refers to his own standing in the world, his political party, and the viewpoints he associates himself with. The purpose of this speech is clearly to convince the audience that his viewpoint is right, without having them question the legitimacy of what he is saying. This clearly shows that he is empathetic and participatory. 

        Finally, the Donal Trump speech is agonistically toned. The entirety of the speech is worded in a competitive fashion. His speech pushes his own personal agenda of accepting nuclear power in the United States, as well as highlighting the differences between Republicans and Democrats. He even plays the victim role, saying that if he were a Democrat he’d be declared much more intelligent than he is. 


    \section{}
        \textbf{For this question, we’ll conduct an experiment. Get on a website that uses algorithms to make recommendations for you (YouTube, Twitter, or Facebook would be good. No porn sites.). Follow its recommended links for 20 or so recommendations. Now write up your results. How did this experience illustrate concepts from our discussion of algorithms, homophilia, AI bias, and social sorting? Your answer should spend a paragraph documenting the experience (what website did you use, what recommendations did it keep feeding you, how did where you stopped relate to where you started?). Next, identify two insights about the course concepts (algorithms, homophilia, AI bias, and social sorting) based on your experience. Here you should make a claim and support it with evidence from your experience.}
        
        I decided to choose Instagram for this experiment. I went onto my “Explore” page, which 
        recommends numerous Instagram posts based on my already pre-existing interests that the 
        algorithm picked up. This is an excellent example for me because I have not used Instagram 
        lately and as such, saw more diversified content appear on my page at the beginning. 
        I decided to click only on basketball-related content, to see how my page would change 
        after a certain amount of time. After every post I clicked on, I refreshed the page and 
        repeated. After only 10 posts, my entire feed was almost completely filled with 
        basketball-related content. 

        This experience clearly shows the homophilic tendencies that are used by Instagram. 
        We defined in the lecture that “Machine learning systems learn from data, rely on patterns, 
        and make decisions” \citep{lesson6}. In this scenario, there is an internal
        machine learning system that is built specifically to create “Filter bubbles”, as we 
        learned in the lecture, which “show you more of the same type of information” \citep{lesson6}. When I decided to click only on the basketball information, the 
        machine learning algorithm learned from my data and used clear patterns (that I was 
        only interested in basketball) in order to place me in a filter bubble that only exposed 
        me to basketball information. If I was a regular user who was not trying to understand 
        the inner workings of the algorithm, this would show me information that was catered 
        only to my liking and thus make it so that I came back to the app more often. Thus, 
        the algorithm is making recommendations based on what it thinks I (and others similar to 
        me) like to consume. 

        The second concept this experience illustrated was that of Instagram’s internal 
        algorithms. Becca Lewis’s paper applies to a very similar scenario to what I saw 
        in this experiment. In the experiment, she speaks about how “YouTube’s recommendation 
        algorithm can lead users down a rabbit hole” \citep{lesson6}. By the end of my 
        experiment, I was solely receiving content related to basketball. As a result, my 
        ntire Instagram feed consisted of material that I allegedly like to consume, allowing 
        me to get trapped in the “endless scroll” that we often see as a criticism of apps 
        such as Instagram. This would make me keep using the application, meaning Instagram 
        gets more data on me. And, as we learned in lecture if an app/service is free “you are 
        the product” \citep{lesson6}. 


    \section{}
        \textbf{ We learned how communication technologies have made it easier for governments and companies to track and surveil our activities. Taking Google and NSA’s data tracking and surveilling practices explain the concept of panopticonism. Next, discuss why these practices are maybe problematic for your privacy and consent rights.}
        
        Google is the largest search engine on the planet. In this day and age, almost everyone 
        across the world uses the product on a daily basis (often multiple times in a day). The 
        company has gotten so large that they now have a presence in multiple businesses (phones, 
        cloud services, even autonomous vehicles) and the dictionary even considers “Google” a verb. 
        What many users don’t know, however, is that Google is a very practical example of digital 
        panopticonism. 

        The definition of panopticonism is “surveillance cultures as an organizing social 
        dynamic and ideology” \citep{lesson6}. This was originally described in the context 
        of a physical jail cell block design in which the cells were organized in an almost 
        circular pattern, with a surveillance tower in the middle. The inmates were never aware 
        of when they were being watched by the guards in the tower, and as such, operate under 
        the pretense that they were always being spied on. This proved to be a very effective 
        form of administration for the jail cell. The digital panopticon regarding Google is a 
        very similar concept, except the “inmates” being spied on are the users who use the search 
        engine while the “guards” are Google themselves. While using the service, Google is 
        constantly tracking and collecting our data in order to sell to the government. 

        We learned in the lecture that all people are categorizable into one of three types: 
        privacy fundamentalists, privacy pragmatists, and privacy unconcerned. Privacy 
        fundamentalists protect “privacy first. An ideological commitment to privacy”. Pragmatists 
        trade “some privacy for some benefit”. Finally, those who are unconcerned believe that 
        the “benefits are far greater than any threats” and are “willing to give lots of information 
        for small benefits” \citep{lesson6}. Those who use Google even though they are 
        aware of the digital panopticonism fall under either the pragmatist or unconcerned category. 
        The products that Google offers are tailored towards ease of use, and as mentioned prior, 
        it has become commonplace in society to rely on Google. 

        However, since Google is an entity that no one using the internet can physically see, 
        the ever-watching eye is not as apparent. If there were a person staring over your 
        shoulder at everything you typed into your computer (and then wrote down whatever you typed), 
        you would act much differently even if you were in the unconcerned group. 
        This is exactly what is happening every time you enter a search into Google’s website! 
        Privacy and consent are therefore non-existent in this scenario, as every user’s private 
        search histories are logged and bookmarked in Google’s databases (and most likely sold for 
        profit!)

        \pagebreak
        \bibliographystyle{apalike}
        \bibliography{myrefs}
        \cite{good2014}
        \cite{lesson4}
        \cite{lesson6}
        \cite{ong1982}
        \cite{hemo2015}
        \cite{timoti1993}
\end{document}


