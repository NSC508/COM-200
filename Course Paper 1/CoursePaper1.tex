\documentclass[a4paper]{article}
\usepackage[utf8]{inputenc}
\usepackage[english]{babel} 
\usepackage{hyperref} 
\usepackage{float}
\usepackage{amsmath} 
\usepackage{graphicx}
\usepackage[colorinlistoftodos]{todonotes} 
\usepackage{tikz}
\usepackage{pdfpages} 
\usepackage{listings}
\usepackage[margin=0.5in]{geometry}
\usepackage{natbib}

\title {
	Course Paper 1
}

\author {
	\normalsize Sathvik Chinta\\\normalsize
    \normalsize Communication 200\\\normalsize
}

\date {
	\color{black} October 14, 2022
}


\begin{document} \maketitle
    \setcitestyle{authoryear,open={(},close={)}}
    \section{}
        \textbf{In lecture, we used this definition of Communication: “A systemic process in which people interact with and through symbols to create and interpret meanings” (Wood, 2017). Provide and discuss an example for each of the four underlined aspects of this definition (systemic; process; symbols; create and interpret meanings).}

		The definition of systemic in this context is "Communication exists within pre-existing webs of context" \citep{wood2017}. An example of this
        is when two people are discussing a tradition from their home culture. The context of the conversation is the culture itself, 
        and the communication is the discussion of the tradition. Because both members of the conversation are from the same culture,
        or at least have a similar understanding of the culture, they can communicate effectively. If one of the members of the
        conversation was from a different culture, the communication would be less effective because they do not exist in the same
        pre-existing web of context. For instance, the thread ceremony in India is a very important tradition that many 
        people participate. If a person from India was talking to a person from the United States about the thread ceremony,
        the communication would be less effective because the person from the United States does not exist in the same web of context.

        The definition of a process in this context is "Communication is ongoing and dynamic". An example of this is when a person 
        visits a doctor. The doctor asks the patient questions about their symptoms, and the patient answers the questions. The doctor
        then asks the patient more questions, and the patient answers the questions. This process continues until the doctor has enough
        information to make a diagnosis. The communication is ongoing and dynamic because the doctor is constantly asking questions
        and the patient is constantly answering the questions. 

        The definition of symbols in this context is "Communication is represented through things and words". Communication is not 
        only limited to words but can be expressed in many different states and contexts. For instance, in today's society, people
        can have an entire conversation by just using emojis. Communication is represented through emojis, and the
        emojis are used to create and interpret meanings. Each emoji conveys a different meaning, and the meanings are interpreted
        by the people who are communicating (this goes back to our idea of pre-existing webs of context). 
        
        The definition of creating and interpreting meanings in this context is "Communication is mutually constructed and affects our 
        social reality". An example of this is when a person is discussing a recently released movie with a friend. The two will 
        go back and forth about the movie, and they will both create and interpret meanings. Communication itself is not 
        constructed if one person is just talking and the other person is not listening. The communication is mutually constructed
        because both people are participating in the conversation. Communication affects our social reality because the
        resulting ideas that stem from the conversation will affect the way that the two people view the movie, and 
        perhaps the way they judge other movies in the future.

  
    \section{}
        \textbf{Select a social media post (e.g., Twitter, Instagram, etc. - setting - public) and describe it in terms of the transmission model of communication. Next, describe the same tweet in terms of the transactional model of communication. Finally, discuss the differences between the two models: what does the transactional model include that the transmission model doesn’t?}

        Firstly, we should define the two models of communication. The transactional model is one in which the sender sends a message
        through a channel to a receiver. There may be encoding and decoding involved, and there may be noise. The biggest 
        distinguishing factor is that transactional models are very linear. 

        Conversely, the transmission model is bi-directional. Both the sender and the receiver are sending and receiving messages
        at the same time. There can be multiple messages, and multiple channels and both will be involved in encoding and decoding.
        There can still be noise and feedback like in the transmission model. 

        With that groundwork laid out, I picked the following tweet: \citep{kobeTweet}

        This tweet is made by the basketball player Kobe Bryant. He is sending a message to his followers, and the message is
        "Continuing to move the game forward @KingJames. Much respect my brother 33644" (some symbols not included). 
        
        Through the transactional model, Bryant is the sender. There is a bit of encoding involved, such as understanding
        the context behind the tweet as well as the meanings of the words and symbols. This tweet was sent directly 
        after LeBron James passed Kobe Bryant on the all-time scoring list. Bryant is sending a message to James, and the
        the message is that he respects James and his accomplishments. The number 33644 is the number of points that 
        LeBron James has scored (Kobe had scored 33643 in his career). 

        Through the transmission model, we still maintain all the info from above. However, we also consider that this is 
        the last tweet that Bryant ever sent. He was killed in a helicopter crash on January 26, 2020. The tweet was sent
        on January 25, 2020. As such, this tweet garnered 2.6 million likes, over 495 thousand retweets, and 
        thousands of replies. The fans of Kobe, and basketball as a whole, all gathered together to express their 
        condolences and respect for Kobe in the comment section of this tweet. 

        Through the transmission model, we were only able to paint a small picture of the tweet. We were only able to see
        the sender and the receiver. Through the transmission model, we were able to see the entire picture. We were able
        to see the sender, the receiver, the fans, the media, and everyone else who was involved in the tweet. We are also 
        able to get the context of the tweet and the emotions that were felt by the people who were involved in the tweet.

    \section{}
        \textbf{Find and discuss an example of post-positive OR interpretive research. The examples should be from within the past five years. You could find these on University websites, through academic journals, and/or popular science outlets. In your answer, you should describe the study (including a link). Next, explain how the research exemplifies a trait, method, and concern of this approach.}
        
        An example of post-positive researhc that I found was an article titled "Black-owned restaurants disproportionately impacted during pandemic" by 
        Kim Eckart \citep{eckart2022} ({https://www.washington.edu/news/2022/08/29/black-owned-restaurants-disproportionately-impacted-during-pandemic/}). The study looks at different restaurants in the United States throughout and in 
        the aftermath of the COVID-19 pandemic. Following the wake of the murder of George Floyd, many large "tech companies 
        began promoting "Black-owned" labelling campaigns to encourage customer support for restaurants and other businesses." \cite{eckart2022}.

        The study used cellphone location data to see the true visitation rates of "Black-owned" restaurants, and the researchers found 
        a very interesting trend. In all the major cities that were observed (Las Vegas, San Fransisco, Detroit, Denver, New York, Dallas, 
        Philadelphia, Los Angeles, Houston, Miami, Seattle, Washington D.C., Chicago, Atlanta, Charlotte, Phoenix, Baltimore, Minneapolis, 
        New Orleans, and Memphis) the researchers observed a sharp decline in visitation numbers after the murder. Then, during the campaign, 
        there was a slight upward trend in the number of visitations, but that slowly fell back down again. 

        Researchers also noticed a difference between restaurants that had the label "Black-owned" and those that did not have a label. 
        Certain cities had greater disparities than others. For instance, New Orleans and Detroit had the greatest disparity, while New York
        had the least. The restauraunts that were marked as not having a label were taken as a control group to explain the drop in 
        visitation rates due to the pandemic when people were not going outside as much. 

        This is a post-positive. The trait that clearly shows this is how it's based in trend data and weekly visitation ratios. We are 
        dealing with numbers, seemingly isolated from the researchers. The method they used to perform this research was using the cellphone 
        visitation data. The concern with this approach is that they are assuming that everyone has a cellphone and will always bring it with them 
        to restauraunts. While they had a very large sample size, we must also consider that some people may have their phones disabled from location 
        settings as well, so they would not be represented in this study. 

    \section{}
        \textbf{Find and discuss an example of rhetorical OR critical research. The examples should be from within the past five years. You could find these on University websites, through academic journals, and/or popular science outlets. In your answer, you should describe the study (including a link). Next, explain how the research exemplifies the traits, methods, and concerns of this approach.}
        
        Critical research concerns powers. It states that power relations fundamentally 
        affect our understanding of reality. 

        An example of critical research that I found was an article titled "Reviving Bruce: negotiating Asian masculinity 
        through Bruce Lee paratexts in Giant Robot and Angry Asian Man" by LeiLani Nishime \citep{nishime2016} 
        ({https://www.tandfonline.com/doi/full/10.1080/15295036.2017.1285420?journalCode=rcsm20}). The study looks at the 
        representation of Bruce Lee which "address and undermine dominant stereotypes while also enabling a distinct 
        “zone of transaction” between Asian American audiences and Lee's films." \citep{nishime2016}. 

        The paper also attempts to explain the "understand the construction of Asian American masculinity 
        at the nexus of hegemonic white masculinity, denigrated African American masculinity, 
        and highly gendered “outsider” and nerd cultures." \citep{nishime2016}.

        The study looks at the representation of Bruce Lee specifically in various paratextual contexts, and 
        explains how they actually enable something called the "zone of transaction" in which powerful media 
        attempts to change the view of Asians in film from "effeminate, asexual, and obsessed with white women" 
        to one that is "still largely asexual, invoked a violent masculinity tempered by an “Eastern” philosophy of 
        bodily mastery and control" \citep{nishime2016}. 

        The article looks at two specific examples in order to show this change. The first example is that of 
        "a Bruce Lee t-shirt manufactured by the Asian American 'zine Giant Robot in the 90" \citep{nishime2016}. 
        The second example comes from 20 years later, and is a "kung fu action figure on the homepage of the influential 
        blog Angry Asian Man" \citep{nishime2016}. The researchers talk about the perpetuation of certain stereotypes
        in the media, and how filmmakers are still pushing that. 

        This article is a critical approach since it brings to light the power relations that are at 
        play in the media. The researchers are looking at the power relations between the media and the
        audience, specifically in the portrayal of Bruce Lee and Asian men in general. 

        The method used in this study is a textual analysis of the paratexts. The concern with this approach is that
        the researchers are using a persuasive interpretation. There is only one side to the story, and the researchers
        are trying to persuade the audience that their interpretation is correct. 

        \pagebreak
        \bibliographystyle{apalike}
        \bibliography{myrefs}
        \cite{wood2017}
        \cite{kobeTweet}
        \cite{eckart2022}
        \cite{nishime2016}
\end{document}


